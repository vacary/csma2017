\documentclass[8pt,red]{beamer}
\usepackage{amsmath,amsthm,amsfonts,amssymb,amsbsy}
\usepackage{stmaryrd} 
\usepackage{subeqnarray}

\usepackage{cases}

\usepackage{pifont}

\usepackage{fancyhdr}

\usepackage{lastpage}


\usepackage{graphicx}
%\usepackage{subfloat}
\usepackage{rotating}
\usepackage{tikz}

\usetikzlibrary{arrows}
\usetikzlibrary{calc}
\usepackage{mathtools}

% \usepackage{algorithm}
% \usepackage{algorithmic}

\usepackage{siunitx}
\usepackage{subfig,float}
\input{macro.tex}

\usepackage{wasysym}



\setbeamertemplate{theorem}[ams style]
\setbeamertemplate{theorems}[numbered]


\input{theme.tex}

\title{Nonsmooth Newton methods for frictional contact problems in flexible multi-body systems}
\author{V. Acary, M. Br\'emond, F. Dubois \\ INRIA Rh\^one--Alpes, Grenoble. \\ LMGC Montpellier}
\date{13\textsuperscript{eme} Colloque National en Calcul des structures \\ [2mm] \today}

\setbeamertemplate 
{footline} 
{\quad\hfill\strut\insertsection\quad--\quad\insertframenumber/\inserttotalframenumber\strut\quad\quad} 

\graphicspath{{../figure/}}

\includeonly{%
%  introduction,
  fc3d,
  %existence,
  numerics,
  comparison
}



%\newtheorem{defn}{Definition}
%\renewcommand{\thedefn}{\arabic{defn}}
% \newtheorem{thm}[defn]{Theorem}
% \newtheorem{corr}[defn]{Corollary}
% \newtheorem{ass}[defn]{Assumption}
% \newtheorem{lem}[defn]{Lemma}
% \newtheorem{rem}[defn]{Remark}
% \newtheorem{hypo}[defn]{Hypotheses}
% \newtheorem{exmp}[defn]{Example}
% \newtheorem{prop}[defn]{Proposition}
\newcommand{\diag}{\mbox{\rm diag}}
\newcommand{\co}{\overline{\mathit{co}}}
\newcommand{\rect}{\overline{\mathit{rect}}}
\newcommand{\newb}{g}

\renewcommand{\tr}[1]{\textcolor{red}{#1}}
\def\nat{{\hbox{\sf \tiny{nat}}}}
\def\ac{{\hbox{\tiny{ac}}}}
\def\mjtwo{{\hbox{\tiny{mj}}}}
\begin{document}

\frame{\titlepage
  \begin{center}
    \includegraphics[height=0.1\textheight]{../figure/logo-inria.png}\\
    \includegraphics[height=0.1\textheight]{../figure/logo-lmgc.jpg}
  \end{center}
  $$ $$
}


%\frame{\tableofcontents}

\begin{frame}
  \frametitle{Objectives}
  
\end{frame}

\include{fc3d}
%\frame{\tableofcontents}

\include{existence}
%\frame{\tableofcontents}


\section{Numerical solution procedure.}
\label{Sec:Numerics}

\subsection{Nonsmooth Equations based methods}
\frame{
  \frametitle{Nonsmooth Equations based methods}
  \begin{block}  {Nonsmooth Newton on $F(r)=0$}

    $$r_{k+1}  =  r_k -  \Phi^{-1}(r_k) (F(r_k)), \quad\quad\Phi(r_k) \in \partial F(r_k)$$
    
   \begin{itemize}
     
   \item Alart--Curnier Formulation~\cite{Alart.Curnier1991}
     \begin{equation}
       \label{eq:AC-1}
       F_\ac(r) \coloneqq  
       \left[
         \begin{array}{l} 
           r_\n - P_{\RR^{n_c}_+}(r_\n - \rho_\n  (Wr+q)_\n) , \\
           r_\t - P_{D(\mu, (r_\n - \rho (Wr+q)_\n)_+)}(r_\t - \rho_\t (Wr+q)_\t   )
         \end{array}
       \right],\quad \rho_\n>0, \rho_\t>0,
     \end{equation}
   \item Jean -- Moreau formulation ~\cite{Jean.Moreau1987,Christensen.Klarbring.ea1998}
     \begin{equation}
       \label{eq:MJ-II}
       F_{\mjtwo}(r) \coloneqq \left[
         \begin{array}{c}
           r_\n - P_{\RR^{n_c}_+}(r_\n - \rho_\n (W r +  q)_\n) \\
           r_\t - P_{D(\mu, (r_{\n})_+)}(r_\t - \rho_\t (Wr+q)_\t   ) 
         \end{array}\right] ,\quad \rho_\n>0, \rho_\t>0.
     \end{equation}
   \item Direct natural map reformulation
     \begin{equation}
       \label{eq:natural-II}
       F_\nat(r) \coloneqq   \left[
         \begin{array}{l} 
           r - P_{K}\left(r  - \rho (Wr+q + g(Wr+q))\right)
         \end{array}\right], \quad \rho >0
     \end{equation}
   \end{itemize}
 MUMPS~\cite{Amestoy.ea_PC2006,Amestoy.ea_SIAMMAA2001} is used for solving linear systems.
 \end{block}
}

\subsection{Matrix block--splitting and projection based algorithms}
\frame{
  \frametitle{Matrix block-splitting and projection based algorithms \cite{Moreau1994,Jean.Touzot1988}}
  \begin{block}{Block splitting algorithm with $W^{\alpha\alpha}\in \RR^3$ (Gauss-Seidel)}
    \begin{equation}\label{EQ:NSGS-local1}
      \left\{ 
        \begin{array}{l}
          {u}^{\alpha}_{i+1} - { W}^{\alpha \alpha} {P}^{\alpha}_{i+1} = {q}^{\alpha} + \displaystyle\sum_{\beta<\alpha}^{~} { W}^{\alpha \beta} {r}^{\beta}_{i+1} + \displaystyle\sum_{\beta >\alpha}^{~} { W}^{\alpha \beta} {r}^{\beta}_{i}\\
          ~\\
          \widehat u^{\alpha}_{i+1} = \left[u_{\n,i+1}^{\alpha}+ \mu^{\alpha}\;||u^{\alpha}_{\t,i+1}||, u^{\alpha}_{\t,i+1}\right]^T \\ \\
          {\bf K}^{\alpha,*} \ni \widehat u^{\alpha}_{i+1}  \perp r^{\alpha}_{i+1} \in {\bf K}^{\alpha} \\
        \end{array} \right.
    \end{equation}
    for all $\alpha \in \{1\ldots m\}$.
  \end{block}
  \begin{block}{One contact point problem}
    \begin{itemize}
    \item closed form solutions
    \item Any solver listed before.  
    \end{itemize}
  \end{block}
  
}
\begin{frame}
  \frametitle{Naming convention}
  \begin{table}
  \centering
  \begin{tabular}{lp{0.7\textwidth}}
    \hline
    {\sf\small NSN-AC-NLS} & Nonsmooth Newton Method using ~(\ref{eq:AC-1}) without line-search \\
    {\sf\small NSN-JM-NLS} & Nonsmooth Newton Method using~(\ref{eq:MJ-II}) without line-search \\
    {\sf\small NSN-NM-NLS} & Nonsmooth Newton Method using~(\ref{eq:natural-II}) without line-search\\
    {\sf\small NSN-AC-NLS-HYBRID} &Method {\sf\small NSN-AC-NLS} with preconditioning with  $100$ iterations of  {\sf\small NSGS-AC} \\
    {\sf\small NSGS-AC} & Gauss--Seidel method with {\sf\small NSN-AC-NLS} as local solver\\
    {\sf\small  NSGS-FP-VI-UPK} & Gauss--Seidel method with fixed point iterations of  $F_\nat(r)-r$\\
    \hline 
  \end{tabular}
  \caption{Naming convention}
  \label{tab:nomenclature}
\end{table}

\begin{block}{Error evaluation}
  \begin{equation}
    \label{eq:stopping-criteria-full}
    \displaystyle\frac{\|F_\nat(r)\|}{\|q\|} < \epsilon,
  \end{equation}
\end{block}

\end{frame}



\subsection{Siconos/Numerics}
\frame{
\frametitle{Siconos/Numerics}
\begin{block}
  {\sc Siconos} Open source software for modelling and simulation of
  nonsmooth systems
\end{block}
\begin{block} {\sc Siconos/Numerics} Collection of C routines to
  solve FC3D problem
  \begin{itemize}
  \item NonSmoothGaussSeidel : VI based projection/splitting
    algorithm
  \item TrescaFixedPoint : fixed point algorithm on Tresca fixed
    point
  \item LocalAlartCurnier : semi--smooth newton method of
    Alart--Curnier formulation
  \item ProximalFixedPoint : proximal point algorithm
  \item VIFixedPointProjection : VI based fixed-point projection
   \item VIExtragradient : VI based extra-gradient method
   \item \ldots
   \end{itemize}
 \end{block}
 \begin{block}{\url{http://siconos.gforge.inria.fr}}
   use and contribute ...
 \end{block}
  
}







%%% Local Variables: 
%%% mode: latex
%%% TeX-master: "s"
%%% End: 

%\frame{\tableofcontents}


\section{Preliminary  Comparisons}
\label{Sec:Comparison}
\subsection{Performance profiles}
\frame{
  \frametitle{Performance profiles~\cite{DolanMore_MP2002}}

  \begin{itemize}
  \item Given a set of problems $\mathcal P$
  \item Given a set of solvers $\mathcal S$  
  \item A performance measure for each problem  with a solver $t_{p,s}$ (cpu time, flops, ...)
  \item Compute the performance ratio
    \begin{equation}
      \label{eq:perf-ratio}
      \tau_{p,s} =    \Frac{t_{p,s}}{\min_{s\in\mathcal S} t_{p,s}} \geq 1
    \end{equation}
  \item Compute the performance profile $\rho_s(\tau) : [1,+\infty]\rightarrow [0,1]$ for each solver $s\in \mathcal S$
    
    \begin{equation}
      \rho_s(\tau) = \Frac{1}{|\mathcal P|}\big|\{p\in \mathcal P\mid \tau_{p,s} \leq \tau    \}\big|\label{eq:perf}
  \end{equation}
  The value of $\rho_s(1)$ is the probability that the solver $s$ will win over the rest of the solvers.
  \end{itemize}
  
  
}
\begin{frame}
  \frametitle{LMGC90 sheared low wall example}
  \begin{figure}[htbp]
  \centering
  \includegraphics[height=0.5\textheight]{../figure/LowWall_FEM.png}
  \caption{A low wall meshes with H8}
  \label{fig:LowWall_FEM}
\end{figure}
\vspace{-0.5cm}
\begin{itemize}
\item H8 FE with Linear elastic behavior :
  $\rho= \num{2000}\si{\kilogram\per\cubic\metre},
  E=\num{2.2e+9}\si{\pascal} ,\nu = 0.2$
\item $\mu= 0.83 $ between block and $\mu =0.53$ between blocks and supports
\item Vertical compression force : $30000 \si{\newton}$ horizontal shear velocity $\num{1e-3}\si{\meter\per\second}$.
\item Sampling of $50$ problems collected in the FCLib with graded difficulty 
\end{itemize}
\end{frame}



\begin{frame}
  \frametitle{Results}
  \begin{figure}[htbp]
    \begin{center}
  \centering 
  \subfloat[Accuracy $\epsilon = 10^{-2}$]{\includegraphics[width=0.55\textwidth]{../figure/LowWall_FEM.1e-2.with_guess/simple/profile-LMGC_LowWall_FEM.pdf}\label{fig:LowWall_FEM.1e-2.simple}}
   \subfloat[Accuracy $\epsilon = 10^{-3}$]{\includegraphics[width=0.55\textwidth]{../figure/LowWall_FEM.1e-3.with_guess/simple/profile-LMGC_LowWall_FEM.pdf}\label{fig:LowWall_FEM.1e-3.simple}}\\
   %\subfloat[Accuracy $\epsilon =  10^{-4}$]{\includegraphics[width=0.49\textwidth]{../figure/LowWall_FEM.1e-4.with_guess/simple/profile-LMGC_LowWall_FEM.pdf}\label{fig:LowWall_FEM.1e-4.simple}}
   %\subfloat[Accuracy $\epsilon = 10^{-6}$]{\includegraphics[width=0.49\textwidth]{../figure/LowWall_FEM.1e-6.with_guess/simple/profile-LMGC_LowWall_FEM.pdf}\label{fig:LowWall_FEM.1e-6.simple}}\\
   % \subfloat[précision $\epsilon = 10^{-8}$]{\includegraphics[width=0.49\textwidth]{figure/LowWall_FEM.1e-8.with_guess/simple/profile-LMGC_LowWall_FEM.pdf}\label{fig:LowWall_FEM.1e-8.simple}}\\
   \includegraphics{../figure/LowWall_FEM.1e-2.with_guess/simple/profile-LMGC_LowWall_FEM_legend.pdf}
 \end{center}
  \caption{Comparison for two different required accuracies}
  \label{fig:LowWall_FEM.simple}
\end{figure}
\end{frame}

\begin{frame}
  \frametitle{Results}
  \begin{figure}[htbp]
  \centering 
  %\subfloat[Accuracy $\epsilon = 10^{-2}$]{\includegraphics[width=0.49\textwidth]{../figure/LowWall_FEM.1e-2.with_guess/simple/profile-LMGC_LowWall_FEM.pdf}\label{fig:LowWall_FEM.1e-2.simple}}
  % \subfloat[Accuracy $\epsilon = 10^{-3}$]{\includegraphics[width=0.49\textwidth]{../figure/LowWall_FEM.1e-3.with_guess/simple/profile-LMGC_LowWall_FEM.pdf}\label{fig:LowWall_FEM.1e-3.simple}}\\
   \subfloat[Accuracy $\epsilon =  10^{-4}$]{\includegraphics[width=0.55\textwidth]{../figure/LowWall_FEM.1e-4.with_guess/simple/profile-LMGC_LowWall_FEM.pdf}\label{fig:LowWall_FEM.1e-4.simple}}
   \subfloat[Accuracy $\epsilon = 10^{-6}$]{\includegraphics[width=0.55\textwidth]{../figure/LowWall_FEM.1e-6.with_guess/simple/profile-LMGC_LowWall_FEM.pdf}\label{fig:LowWall_FEM.1e-6.simple}}\\
   % \subfloat[précision $\epsilon = 10^{-8}$]{\includegraphics[width=0.49\textwidth]{figure/LowWall_FEM.1e-8.with_guess/simple/profile-LMGC_LowWall_FEM.pdf}\label{fig:LowWall_FEM.1e-8.simple}}\\
   \includegraphics{../figure/LowWall_FEM.1e-2.with_guess/simple/profile-LMGC_LowWall_FEM_legend.pdf}
  \caption{Comparison for two different required accuracies}
  \label{fig:LowWall_FEM.simple}
\end{figure}
\end{frame}
%%% Local Variables:
%%% mode: latex
%%% TeX-master: "s"
%%% End:










\section{Conclusions \& Perspectives}
\frame{
\frametitle{Conclusions \& Perspectives}
\begin{block}{Conclusions}
  \begin{enumerate}
  \item For relatively tight accuracy, nonsmooth Newton methods outperform first order iterative method.
  \item  {\sf\small NSN-AC-NLS-HYBRID} is the most efficient method
  \item First order iterative methods are interesting for low accuracy, but are not able to reach tight accuracy,
  \end{enumerate}
\end{block}
\vskip-3mm
\begin{block}{Perspectives}
  \begin{enumerate}
  \item Evaluate the interest to transform rigid model into flexible ones.
  \item Study the possibility to take into account the possible nonlinear bulk behavior in the Newton loop
  \item HPC and scalability  of nonsmooth Newton techniques using MUMPS 
  \item Continue to set up a collection of benchmarks  \ding{220} FCLIB
  \end{enumerate}
\end{block}

}

\subsection{FCLIB : a collection of discrete 3D Frictional Contact (FC) problems}

\frame{
  \frametitle{FCLIB : a collection of discrete 3D Frictional Contact (FC) problems}
  Our inspiration: MCPLIB or CUTEst
  \begin{block}
    {What is FCLIB ?}
    \begin{itemize}
    \item A open source collection of Frictional Contact (FC) problems
      stored in a specific HDF5 format 
    \item A open source light implementation of Input/Output functions
      in C Language to read and write problems (Python and Matlab coming soon)
    \end{itemize}
  \end{block}


  \begin{block}
    {Goals of the project}
    Provide a standard framework for testing available and new algorithms for solving discrete frictional contact problems share common formulations of problems in order to exchange data
\end{block}
\begin{block}
    {Call for contribution}
    \alert{\url{http://fclib.gforge.inria.fr}}
  \end{block}
}

\frame
{
\centerline{\textcolor{red}{ Thank you for your attention.}}
}

 
\def\newblock{}
{\scriptsize
\bibliographystyle{plainnat}
\bibliography{../Report/biblio,./Biblio/NonSmooth,./Biblio/Contact}
}

%\include{PWL}


\end{document}

%%% Local Variables: 
%%% mode: latex
%%% TeX-master: t
%%% End: 
